%% LyX 2.0.3 created this file.  For more info, see http://www.lyx.org/.
%% Do not edit unless you really know what you are doing.
\documentclass[10pt,british,english]{article}
\usepackage[latin9]{inputenc}
\usepackage[a4paper]{geometry}
\geometry{verbose,tmargin=25mm,bmargin=25mm,lmargin=40mm,rmargin=10mm}
\usepackage{color}
\usepackage{babel}
\usepackage{array}
\usepackage{multirow}
\usepackage{amsmath}
\usepackage{setspace}
\doublespacing
\usepackage[unicode=true,pdfusetitle,
 bookmarks=true,bookmarksnumbered=false,bookmarksopen=false,
 breaklinks=false,pdfborder={0 0 1},backref=section,colorlinks=true]
 {hyperref}
\hypersetup{
 linkcolor=black,citecolor=black,filecolor=black,urlcolor=black}
\usepackage{breakurl}

\usepackage[cm]{fullpage}
\usepackage{acronym}

\begin{document}
\selectlanguage{british}%
\begin{abstract}
TODO : Abstract\tableofcontents{}

\cleardoublepage{}
\end{abstract}

\section{Introduction}


\section{Literature Review}


\section{Computational Methods}


\subsection{Potential Field Force Navigation\label{sub:Potential-Field-Force}}

For this project, it was decided that a potential field force navigation
technique would be used to control the individual (as described in
\cite{intelligentAlgorithmPathPlanning}. The field is produced by
taking a collection of objects and projecting a field around them
of a given shape. This field can either add to or subtract from the
total field potential surrounding it. The total field can then be
calculated with equation \ref{eq:fieldSummation}.

\begin{equation}
P(x,y)=\sum_{i=0}^{N-1}p_{i}\left(x,y\right)\label{eq:fieldSummation}
\end{equation}


\begin{doublespace}
\noindent Where:
\end{doublespace}

$P\left(x,y\right)$ is the total field potential at point $\left(x,y\right)$.

$p_{i}\left(x,y\right)$ is the potential produced by object $i$
at a point.

$N$ is the number of field producing objects.

The direction that the robot will move in is then determined by the
gradient of the field at the current point, such that the robot moves
down the field potential. The force vector (which controls the velocity
of the robot) is given by equation \ref{eq:forceSummation}.

\begin{equation}
\boldsymbol{F}(x,y)=-\left(\frac{{\partial P\left(x,y\right)}}{\partial x}+\frac{{\partial P\left(x,y\right)}}{\partial y}j\right)\label{eq:forceSummation}
\end{equation}


If the fields can be differentiated over the entire field, this can
be calculated as in equation \ref{eq:forceDifferentiation}.

\begin{equation}
\boldsymbol{F}(x,y)=-\left(\sum_{i=0}^{N-1}\frac{\partial p_{i}\left(x,y\right)}{\partial x}+\frac{\partial p_{i}\left(x,y\right)}{\partial y}j\right)\label{eq:forceDifferentiation}
\end{equation}


However, if one or more of the fields cannot be differentiated, the
force vector must be calculated discretely. This is achieved using
equation .

\begin{equation}
\boldsymbol{F}(x,y)=\left(P\left(x-1,y\right)-P\left(x+1,y\right)\right)+\left(P\left(x,y-1\right)-P\left(x,y+1\right)\right)j
\end{equation}


The negative gradient is used to ensure the robot moves towards lower
field potentials. The technique would work equally well if it was
attracted to higher field potentials, provided all the equations were
suitably inverted. 

For the initial control technique, the velocity of the robot is proportional
to the force vector. Other techniques are discussed by \cite{intelligentAlgorithmPathPlanning},
but will be experimented on later. 

Navigation of the game field can now be achieved by manipulating the
attractive and repulsive fields to guide the robot to a target. Computational
power allowing, each robot would have a separate field that guides
it based on the goals specific to its role. 


\subsection{Potential Field Shapes}

Attractive fields are used to designate areas which the robot should
be headed for. In order to ensure that the robot always heads towards
them, it is important that their effect is felt across the entire
field. However, they must not be so strong that they overwhelm the
short distance effects of the repulsive fields, but strong enough
to govern the robot's overall destination.

Repulsive fields are used to alter the route of the robot when heading
for these targets. For example, such fields are used to get the robot
to avoid collisions, as well as to control where the robot intercepts
the ball. These fields typically only act over a short distance, as
they should only have an effect when they are required. They need
to be strong enough to overwhelm the attractive fields at short range.

The field shapes experimented with are described in the following
sections, in order to determine the best combination for each scenario.
As the basic fields in use are axisymmetric in shape (or begin that
way), the fields are best described using polar coordinates. The coordinates
are defined as in figure \ref{fig:polarCoordinateFrame}, and are
always centred on the field producing object. This frame of reference
can then be translated to the Cartesian coordinates in use elsewhere
to allow the total field at a point to be calculated.


\subsubsection{Basic Attractive Field\label{sub:Basic-Attractive-Field}}

The field suggested by \cite{intelligentAlgorithmPathPlanning} for
the attractive point is described with equation \ref{eq:quadraticBallField}.

\begin{equation}
P(r,\theta)=\frac{1}{2}k_{attr}r^{2}\label{eq:quadraticBallField}
\end{equation}


Where $k_{attr}$ is the attractive weight assigned to the ball.

This produces an attractive force that is linearly proportional to
the distance to the object (in this case the ball). This is a very
useful field for a route finder (as described in the paper), where
it is desirable for the the robot to stop when it reaches the destination,
and so it needs to slow down as it approaches the target. However,
in this application it is more useful for the robot to make a rapid
approach and not slow down after arrival, as this will hit the ball
away from the robot. 

In order to overcome this problem, the field was simplified to equation
\ref{eq:conicBallField}.

\begin{equation}
P(r,\theta)=\frac{1}{2}k_{attr}r\label{eq:conicBallField}
\end{equation}


This field produces a constant force of $k_{attr}$ acting towards
the centre of the ball, which will hold the robot against the ball
(as far as possible) as the ball and robot move around. It doesn't
provide any guidance once the ball has been reached, and so this will
need to be provided by an additional field once the intercept is complete.


\subsubsection{Basic Repulsive Field\label{sub:Basic-Repulsive-Field}}

One of the rules of the game is that the robot cannot intentionally
collide with an opposing robot, unless they are in possession of the
ball \cite{simurosotSim}. This means that the robot must actively
avoid the other robots. It is also advantageous to be able to avoid
the robots own team, as this prevents the robots getting into a position
where they cannot move as they are fighting against each other.

A apparently simple solution would be to surround all the obstacles
with a region of potential that is substantially higher than the surrounding
area, as described by equation \ref{eq:circleRepulseField}.

\begin{equation}
P\left(r,\theta\right)=\begin{cases}
k_{repulse} & r<R_{0}\\
0 & r\geq R_{0}
\end{cases}\label{eq:circleRepulseField}
\end{equation}


Where $k_{repulse}$is the repulsive potential and $R_{0}$is the
radius of the circle. This produces a field as illustrated in figure
\ref{fig:circleField}.

In theory, the robot would avoid this circle, as the algorithm should
not have the robot move into a region of high potential. However,
this does not work, as the controller actually responds to the gradient
of the field (shown in figure \ref{fig:circleFieldGradient}). As
shown in the figure, this type of field produces very high force,
but only in a very small region. Once inside the repulsive circle,
the gradient returns again to zero, and so does not effect the motion
of the robot. As the robot has inertia, it is very likely that it
pass through the region of high force and then leave it before it's
velocity has altered sufficiently to avoid the obstacle.

In order to maximise the repulsive effect of the field, the force
must be exerted over a large area. This can be achieved with a field
that gradually builds in strength as it nears the obstacle. The simplest
example is defined by equation .

\[
P\left(r,\theta\right)=\begin{cases}
k_{repulse}\cdot r & r<R_{0}\\
0 & r\geq R_{0}
\end{cases}
\]


This field will create a constant force that acts radially away from
the obstacle, much like the attractive force described in section
\ref{sub:Basic-Attractive-Field}, but over a restricted area (as
illustrated in figure \ref{fig:conicRepelField} and \ref{fig:ConicRepelFieldGradient}).
If the target is behind the obstacle, and the force is exactly strong
enough, it will cause the robot to stop in the repulsive field. If
the force is too small, it will only slow the robot down, and if the
force is too strong it will cause the robot to move clear of the field.
If the target hasn't moved significantly, the robot will then approach
the repulsive field again, only to be repelled once more. This will
result in the robot repeatedly 'bouncing' off of the field. The robot
will become stuck if the target does not move, but it is thought that
the rapidly changing environment the controller will be working in
will mean that this is not a significant issue.

While the constant force will work, provided it is strong enough,
the case when it is too strong and causes 'bouncing' is undesirable
as the motion of the robot will become erratic. A better solution
would be one where the force gradually increases as it approaches
the target. It is also desirable to have a field that naturally decays
to a small level at a distance without relying on conditional operations.
Two field shapes which meet this specification are a direct inverse
proportionality, or the Gaussian function (described by equations
\ref{eq:inverseRepulseField} and \ref{eq:gaussianRepulseField} respectively).

\begin{equation}
P\left(r,\theta\right)=\frac{k_{repulse}}{r}\label{eq:inverseRepulseField}
\end{equation}


\begin{equation}
P\left(r,\theta\right)=k_{repulse}\cdot e^{\left(\frac{r^{2}}{2\sigma_{repulse}}\right)}\label{eq:gaussianRepulseField}
\end{equation}


Where $\sigma_{repulse}$ controls the width of the field.

For this purpose, the Gaussian function was selected, because it was
initially believed that a flatter region in the centre of the field
would be preferable over an infinite field potential (see section
TODO). This produces a field as shown in figure \ref{fig:gaussianField}.
As the gradient image shows, this produces a region of rapidly increasing
force around the object, with no discontinuities that produce disruptive
motion at the edge of the field.


\subsubsection{Shaped Approach Guiding Field\label{sub:Shaped-Approach-Guiding}}

As the robot has no means of holding on to the ball, it is not possible
for the robots to turn more than a small amount when in possession
of the ball without losing it. This makes it particularly important
that the robot approaches the ball from the correct side. This can
be achieved by positioning a region around the ball that guides the
robot to the correct position.

The initial attempt at this placed a specially shaped field around
the ball, which was developed by rotating a Gaussian function along
a circle around the ball, with it's height proportional to the angle
from the desired access direction. This is described by the equation
\ref{eq:wrappedGaussian} and shown in figure \ref{fig:wrappedGaussianField}.

\begin{equation}
P\left(r,\theta\right)=e^{\frac{-\left(r-r_{0}\right)^{2}}{2\sigma_{repulse}}}\cdot\left|\frac{\theta}{\pi}\right|\label{eq:wrappedGaussian}
\end{equation}


\[
-\pi\leq\theta\leq\pi
\]


This field initially looked promising, as it contains the desired
slope towards a specific approach angle and repels away from every
other angle. On testing, however, it was noticed that, because the
field is at a higher potential than it's surroundings, the edges of
the field force the robot radially away from the ball instead of towards
the target angle, and so the robot just stops at the edge of the field.

The field was then modified to be at a lower potential than it's surroundings,
as shown in equation \ref{eq:insetWrappedGaussian}.

\begin{equation}
P\left(r,\theta\right)=e^{\frac{-\left(r-r_{0}\right)^{2}}{2\sigma_{repulse}}}\cdot\left|\frac{\pi-\theta}{\pi}\right|\label{eq:insetWrappedGaussian}
\end{equation}


This removed the previous radial force. However, it was determined
that a field wide enough to be of any use in directing the robot had
to have a radius that caused the field to strongly interfere with
motion elsewhere in the playing field. When other robots were introduced
into the field, even in their starting positions, the robot was unable
to approach the ball without a collision with another player.

Other attempts were made with different shaped fields around the ball
(for example one field resembled a helter-skelter slide, which produced
a constant force to guide the ball in when within a certain radial
region), but all resulted in either an large field that caused too
much long-distance interference, or forced the robot radially away
from the robot. While a field could probably be found that achieved
what was desired, it was determined that the search would be too time-consuming
with the lack of guiding information.


\subsubsection{Paired Source Approach Guidance Field}

As discussed in section \ref{sub:Shaped-Approach-Guiding}, a field
was required to guide robot's the approach to ball. As a single field
did not function as desired, it was decided that a pair of fields
would be used.

A basic attractive field (see section \ref{sub:Basic-Attractive-Field})
was positioned on the desired approach side of the ball (offset by
a number of inches in the $x$ axis) and a basic repulsive field (see
section \ref{sub:Basic-Repulsive-Field}) was placed in the symmetrically
opposite position. This produces the overall field shown in figure
\ref{fig:pairedApproachField}. 

This technique immediatly showed positive results, with the robot
moving onto the attractive point while avoiding the ball if it approached
from the wrong side. This is achieved because the repulsive field
forces the robot away from any approach vectors which would pass too
close to the ball when the robot is on the opposite side, and the
moved attraction point prevents the robot from colliding with the
ball from the side

TODO : Discuss modifying the field to place the sources in the line
of motion instead of the along the x-axis.

TODO : Discuss any other modifications to the process.

TODO : Consider including a video showing this in action, or perhaps
include it in the results section.

When this field is in use, the robot is not attracted to the actual
position of the ball, and so this field will not allow the robot to
guide it to the goal. However, if a new set of field configurations
are used once this robot is in position, this limitation can be overcome.


\subsection{Field Calculations}

TODO : use acronym package

TODO : use siunitx package

In order to determine the direction the robot should move in, the
field needs to be calculated at four points, as described in section
\ref{sub:Potential-Field-Force}. Even for the most complex fields
in use, this is not particularly computationally challenging. However,
the efficiency of the code could be improved by vectorising the data
and taking advantage of the Single Instruction Multiple Data (SIMD)
instructions available on most modern CPUs to calculate all four points
simultaneously. This would allow the field calculations to take less
time, resulting in more time available per time-step to perform other
operations.

In addition, it is useful to render the entire field as an image,
so that it can be considered for immediate debugging purposes. Given
that the playing field is approximately 88 inches by 72 inches, and
is being considered at $\frac{1}{10}$inch scale, this gives 633,600
data points to consider. This size of dataset will require a large
amount of CPU cycles to calculate, even if the individual data point's
requirements are relatively modest. With further delays introduced
by inter-process communication (it is not possible to alter the simulator
to produce the image locally), as well as other delays and overheads
introduced by the operating system, it quickly becomes challenging
to render the field in real-time.

The calculation times for both the entire field and a set of four
points using the OpenCl code on both the CPU and GPU are shown in
table \ref{tab:Field-Strenth-Calculation}. These clearly show that
the calculation of the entire field is best done on the GPU, where
the acceleration from the massively parallel computation structure
masks the additional overheads. The four points, however, are best
done on the CPU, where it does not suffer from the significantly larger
memory transfer times which affect GPU operations.

\begin{table}
\centering%
\begin{tabular}{|c|>{\centering}m{2cm}|>{\centering}p{2cm}|>{\centering}p{3cm}|>{\centering}m{2cm}|}
\hline 
\multirow{2}{*}{Task} & \multirow{2}{2cm}{\begin{singlespace}
\centering{}Computation Platform\end{singlespace}
} & \multirow{2}{2cm}{\begin{singlespace}
\centering{}Execution Time (us)\end{singlespace}
} & \multirow{2}{3cm}{\begin{singlespace}
\centering{}Memory Transfer Time (us)\end{singlespace}
} & \multirow{2}{2cm}{\begin{singlespace}
\centering{}Total Time (us)\end{singlespace}
}\tabularnewline
 &  &  &  & \tabularnewline
\hline 
\multirow{2}{*}{Entire field} & \begin{singlespace}
CPU\end{singlespace}
 & \begin{singlespace}
23500\end{singlespace}
 & \begin{onehalfspace}
479\end{onehalfspace}
 & \begin{singlespace}
24100\end{singlespace}
\tabularnewline
\cline{2-5} 
 & \begin{singlespace}
GPU\end{singlespace}
 & \begin{singlespace}
2050\end{singlespace}
 & \begin{singlespace}
2600\end{singlespace}
 & \begin{singlespace}
8611\end{singlespace}
\tabularnewline
\hline 
\multirow{2}{*}{Four data points} & \begin{singlespace}
CPU\end{singlespace}
 & \begin{singlespace}
14.4\end{singlespace}
 & \begin{singlespace}
0.342\end{singlespace}
 & \begin{singlespace}
60.0\end{singlespace}
\tabularnewline
\cline{2-5} 
 & \begin{singlespace}
GPU\end{singlespace}
 & \begin{singlespace}
18.8\end{singlespace}
 & \begin{singlespace}
1040\end{singlespace}
 & \begin{singlespace}
4390\end{singlespace}
\tabularnewline
\hline 
\end{tabular}

\begin{singlespace}
In this test, overheads include transfering the inital data to the
platform and setting up the platform before the calculation, but not
one-time-only intialisation done by the platform.
\end{singlespace}

\caption{\selectlanguage{english}%
Field Strenth Calculation Times\label{tab:Field-Strenth-Calculation}\selectlanguage{british}%
}
\end{table}



\subsubsection{Design Philosophy}

The code (shown in appendix \ref{sub:OpenCL-Kernels}) is written
in an attempt to take advantage of the parallel nature of it's execution.
Each kernel (a high-level function run either on the CPU or GPU) is
designed to work on an individual data point, and then the kernel
is called multiple times by the platform, once for each data point.
As the order of execution cannot be guaranteed, each kernel is written
such that it doesn't depend on the values produced for other data
points.

To simplify the code, the platform is configured to run the kernel
over a 2D space that represents the field. Each instance of the kernel
is then given an ID in each dimension by the computation platform,
which is used to determine the coordinates of the data point it should
be working on. This means that each instance can operate in isolation
without any knowledge of what work has already been done. The results
are then stored in a shared array (at an index determine by the coordinates)
which is returned to the host program when the operation has finished.
Where consecutive functions are required, a series of kernels are
queued in turn, and the memory is only returned when the queue is
complete (which represents a significant saving with the GPU code,
as GPU to host transfers are relatively computationally expensive).

For the first scenario, where only a limited number of points are
required, the code is executed on the CPU in parallel, as the advantages
of using the more powerful GPU are lost in the initial set-up delays
which are not incurred when working on the CPU. The GPU is then used
when the entire field is desired, when the accelerated calculations
justify the set-up costs (see TODO RESULTS for exact calculations).


\subsubsection{GPU Optimisations}

The architecture of a GPU presents different optimisation challenges
to that of CPU. In particular, the number of concurrent operations
the GPU can process is limited by the resource usage of the code.

In order to calculate the Gaussian function, a function to calculate
a power of $e$ is needed. Initially, the built in OpenCL function
was used to calculate this. This function is defined by the OpenCL
standard to have a specific accuracy, which is constant on any hardware.
The implementation of this on the hardware in use proved to use a
large number of General Purpose Registers (GPRs), imposing a limit
on the number of concurrent kernel instances that could run and limiting
the speed of the code.

The OpenCL standard also provides for a set of so-called 'native'
functions, are produced specifically for the hardware in use, and
which are often more efficient than the general implementation (for
example, it sometimes maps to a single instruction that performs the
function). However, the accuracy of the function is implementation
specific, and so could change from computer to computer \cite{openCl11Spec}.
In this case, the native exponential function utilitises far fewer
GPRs, removing the resource pressure previously experienced.

Additionally, the OpenCL specification provides geometric functions
such as vector distance and vector length to complement it's vector
data types. These have been used as far as possible, as they can also
allow the compiler to perform hardware specific optimisations (and
again they can sometimes map directly to individual instructions on
the hardware.\cite{openCl11Spec}

TODO : discuss any other optimisations


\section{Software Produced}

Two peices of software have been produced:
\begin{itemize}
\item The game playing strategy file
\item The field renderer
\end{itemize}
Only the strategy file is required to play the game, as the field
renderer is only used to debug the potential fields.


\subsection{Strategy File}

The strategy file is a standard Microsoft Windows DLL file which implements
three functions defined by the simulator:
\begin{itemize}
\item Create - Performs the initial setup for the strategy
\item Strategy - Called on every simulator cycle to control the robots
\item Destroy - Intended to perform the clean-up required for the strategy.
This does not appear to be called by the simulator at this time, and
so has been left as an empty function.
\end{itemize}
TODO : Discuss final strategy implementation


\subsection{Field Renderer}

TODO : tie up descriptions to function names

The field renderer provides a near realtime view of the potential
field and the field gradients at run time to allow easier debugging
of the field calculations. The program is implemented using the MS
.Net Framework and the Windows Presentation Foundation. The OpenCL
code is then executed using the Cloo library, which allows the program's
managed code to access the unmanaged functions required to use OpenCL.

The program receives a status from the simulator using a named pipe
for which it acts as the server. The binary data received is intepretted
into a structure to match the one used in the strategy file, and is
stored as the latest environment. This occurs at the end of every
simulator cycle, and is run in a seperate execution thread to the
rest of the program. This allows the simulator to continue after transmission
is complete without waiting for the renderer to work with the environment.
Every time the status report is received, it triggers a rendering
cycle, unless a rendering cycle is already underway. If a report arrives
while a rendering cycle is unfinished, it is discarded. This ensures
that the rendered images are as up to date as possible.

The rendering process is also executed in it's own thread, ensuring
that it does not interfere with the user interface. If this was not
done, the program would to be unresponsive and the rendered images
would never be displayed to the user. The process creates a copy of
the latest status report, which ensures that the process is not effected
by the concurrently running communication with the simulator. The
code then performs the same process as the strategy file, using the
OpenCL field calculation code (this time executed on the GPU) to calculate
every data point in the field. Further OpenCL code is then used to
calculate the magnitude of the gradient across the entire field, and
more is used to transform both sets of data into grayscale bitmaps.
The produced bitmaps are then passed to the user interface thread,
which then displays the images to the user.


\section{Discussion}


\section{Future Work}

\appendix
\cleardoublepage{}


\section{Code Listings \label{sec:Code-Listings}}


\subsection{OpenCL Kernels\label{sub:OpenCL-Kernels}}


\subsection{Intercept Strategy File}

\cleardoublepage{}

\bibliographystyle{plain}
\bibliography{References/references}
\selectlanguage{english}%

\end{document}
