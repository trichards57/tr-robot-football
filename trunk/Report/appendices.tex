\appendix


\section{OpenCL Kernels\label{sub:OpenCL-Kernels}}

The following listings contain a sub-set of the total code produced. The full
code listings can be found on the included CD, along with the required build
instructions and any additional libraries.

\subsection{Basic Field Shapes}

\lstset{language=C, morekeywords={float2, inline, __constant, __global, int2,
size_t, __kernel},basicstyle=\small\ttfamily}

\subsubsection{Gaussian Repulsive Field\label{app:gaussianRepulsive}}

\begin{lstlisting}

#define BALL_WEIGHT 10
#define OBSTACLE_WEIGHT 100
#define OBSTACLE_SIGMA 5

inline float basicRepel(float2 realPos, float2 repulser)
{
  float2 diff = repulser - realPos;
  return OBSTACLE_WEIGHT * native_exp(-((diff.x*diff.x) + (diff.y*diff.y)) /
  (2*OBSTACLE_SIGMA)); 
}
\end{lstlisting}

\subsubsection{Stretched Gaussian Repulsive
Field\label{app:stretchedGaussianRepulsive}}

\begin{lstlisting}
inline float stretchedBasicRepel(float2 realPos, float2 repulser)
{
  float2 diff = repulser - realPos;
  return OBSTACLE_WEIGHT * native_exp(-((diff.x*diff.x) / (2* OBSTACLE_SIGMA *
  6) + (diff.y*diff.y)/ (2*OBSTACLE_SIGMA)) ); 
}
\end{lstlisting}

\subsection{Field Calculations}

\subsubsection{Initial Approach Field\label{app:initialApproachField}}

\begin{lstlisting}
// Calculates the field at a given point
float fieldAtPoint(float2 realPos, float2 ball, float2 ballVelocity, 
                   __constant float2 *basicRepulsers)
{
  float2 posShift = normalize(ballVelocity)*-7;
  float2 attractBall = ball - posShift;
  float2 repelBall = ball + posShift;
  float dist = distance(attractBall, realPos);
  float attractField = BALL_WEIGHT * dist;

  float repField;

  if (length(posShift) > 1.0f)
    repField += basicRepel(realPos, repelBall);

  for (int i = 0; i < 10; i++)
  {
    repField += basicRepel(realPos, basicRepulsers[i]);
  }

  return attractField + repField;
}

// Used to calculate the entire field
__kernel void main(float2 ball, float fieldResolution, 
                   __constant float2 *basicRepulsers, __global float * out,
                   float2 ballVelocity) 
{
  int2 gridPos = (int2)(get_global_id(0),get_global_id(1));

  size_t index = gridPos.x + gridPos.y * get_global_size(0);

  float2 realPos = convert_float2(gridPos) * fieldResolution;

  float res = fieldAtPoint(realPos, ball, ballVelocity, basicRepulsers);

  out[index] = res;
}

// Used to calculate the field at a set of given points
__kernel void fieldAtPoints(float2 ball, __constant float2 *fieldPoints, 
                            __constant float2 *basicRepulsers, 
                            __global float * out, float2 ballVelocity) 
{
  size_t pointId = get_global_id(0);

  out[pointId] = fieldAtPoint(fieldPoints[pointId], ball, ballVelocity,
  basicRepulsers); 
}
\end{lstlisting}

\subsubsection{Ball Possession Field \label{app:possessionField}}

\begin{lstlisting}
// Calculates the possession field at a given point
float possessionFieldAtPoint(float2 realPos, float2 ball, 
                             __constant float2 *basicRepulsers, 
                             float2 goalTarget) 
{
  float2 vectorToGoal = normalize(goalTarget - realPos)*5;
  float2 attractBall = ball + vectorToGoal;

  float dist = distance(attractBall, realPos);
  float attractField = BALL_WEIGHT * dist;

  float repField = 0;

  for (int i = 0; i < 10; i++)
  {
    repField += basicRepel(realPos, basicRepulsers[i]);
  }

  return attractField + repField;
}

// Used to calculate the entire possession field
__kernel void possessionMain(float2 ball, float2 goalTarget, 
                             float fieldResolution, 
                             __constant float2 *basicRepulsers, 
                             __global float * out) 
 { 
  int2 gridPos = (int2)(get_global_id(0),get_global_id(1));

  size_t index = gridPos.x + gridPos.y * get_global_size(0);

  float2 realPos = convert_float2(gridPos) * fieldResolution;

  float res = possessionFieldAtPoint(realPos, ball, basicRepulsers, goalTarget);

  out[index] = res;
}

// Used to calculate the possession field at a given point
__kernel void possessionFieldAtPoints(float2 ball, float2 goalTarget, 
                                      __constant float2 *fieldPoints, 
                                      __constant float2 *basicRepulsers, 
                                      __global float * out) 
{
  size_t pointId = get_global_id(0);

  out[pointId] = possessionFieldAtPoint(fieldPoints[pointId], ball,
  basicRepulsers, goalTarget); 
}
\end{lstlisting}

\section{Code Listings \label{sec:Code-Listings}}

The following listings contains a sub-set of the total code produced.  The full
code listings can be found on the included CD, along with the required build
instructions and any additional libraries.

\subsection{Intercept Strategy File}



\cleardoublepage{}

\bibliographystyle{plain}
\bibliography{References/references}